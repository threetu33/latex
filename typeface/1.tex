\documentclass[10pt]{article}
%设置normalsize的大小为10pt(一般只有10,11,12pt)
\usepackage{ctex}

\begin{document}
    %字体设置(罗马字体,无衬线字体,打字机字体)
    \textrm{Roman Family}
    \textsf{Sans Serif Family}
    \texttt{Typewriter Family}


    %声明后续字体为罗马字体
    \rmfamily Roman Family {\sffamily Sans Serif Family} {\ttfamily Typewriter Family}

    \rmfamily who are you?
    are you sure?

    \sffamily who are you?
    
    are you sure?

    %字体系列设置(粗细,宽度)
    \textmd{Medium Series}
    \textbf{Boldface Series}
    {\mdseries Medium Series}
    {\bfseries Boldface Series}

    %字体形状(直立,斜体,伪斜体,小型大写)
    \textup{Upright Shape}
    \textit{Italic Shape}
    \textsl{Slanted Shape}
    % \textsc{Small Caps Shape}

    {\upshape Upright Shape}
    {\itshape Italic Shape}
    {\slshape Slanted Shape}
    % {\scshape Small Caps Shape}

    % 中文字体
    {\songti 宋体}
    {\heiti 黑体}
    {\fangsong 仿宋}
    {\kaishu 楷书}

    % 中文字体的\textbf{粗体}与\textit{斜体}

    %字体大小
    {\tiny             hello}\\
    {\scriptsize       hello}\\
    {\footnotesize     hello}\\
    {\small            hello}\\
    {\normalsize       hello}\\
    {\large            hello}\\
    {\Large            hello}\\
    {\LARGE            hello}\\
    {\huge             hello}\\
    {\Huge             hello}\\

    % 中文字号设置命令
    \zihao{3} 你好!
    % -3表示小三号
\end{document}