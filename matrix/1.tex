\documentclass{ctexart}
\usepackage{amsmath}

\begin{document}
    \[
        \begin{matrix}
            0 & 1 \\
            1 & 0
        \end{matrix}
    \]
    \[
    \begin{pmatrix}
        0 & 1 \\
        1 & 0
    \end{pmatrix}
    \]
    \[
    \begin{bmatrix}
        0 & 1 \\
        1 & 0
    \end{bmatrix}
    \]
    \[
    \begin{Bmatrix}
        0 & 1 \\
        1 & 0
    \end{Bmatrix}
    \]
    \[
    \begin{vmatrix}
        0 & 1 \\
        1 & 0
    \end{vmatrix}
    \]
    \[
    \begin{Vmatrix}
        0 & 1 \\
        1 & 0
    \end{Vmatrix}
    \]


    %常用省略号\dots,\vdots,\ddots
    \[
    A = \begin{pmatrix}
        a_{11}^2 & a_{12}^2 & a_{13}^2 \\
        0 & a_{22} & a_{23} \\
        0 & 0 & a_{33}
    \end{pmatrix}  
    \]

    \[
        A=\begin{bmatrix}
            a_{11} & \dots & a_{1n} \\
             & \ddots & \vdots \\
            0 & & a_{nn} 
        \end{bmatrix}\]
    
    % 分块矩阵(嵌套矩阵)
    \[
        \begin{pmatrix}
            \begin{matrix}
                1&0\\0&1
            \end{matrix} & \text{\Large 0} \\
            \text{\Large 0} & \begin{matrix}
                1&0\\0&-1
            \end{matrix}
        \end{pmatrix}\]

    % 行内小矩阵(smallmatrix)环境
    复数$z=(x,y)$ 也可以用矩阵
    \begin{math}
        \left(\begin{smallmatrix}
            x & -y \\ y & x
        \end{smallmatrix}\right)
    \end{math}
    
    % 三角矩阵
    \[\begin{pmatrix}
        a_{11} & a_{12} & \dots & a_{1n} \\
        & a_{22} & \dots & a_{2n} \\
        & & \ddots & \vdots \\
        \multicolumn{2}{c}{\raisebox{1.3ex}[0pt]{\Huge 0}} & & a_{nn}
    \end{pmatrix}\]
\end{document}