\documentclass{article}
\usepackage{ctex}

\begin{document}
    \section{简介}
    \LaTeX{}将排版内容分为文本模式和数学模式,
    文本模式用于普通文本排版,数学模式用于数学公式排版。
    
    \section{行内公式}
    \subsection{美元符号}
    交换律是$a+b=b+a$,如$1+2=2+1=3$。
    \subsection{小括号}
    交换律是\(a+b=b+a\),如\(1+2=2+1=3\)。
    \begin{math}
        a+b=b+a
    \end{math},如
    \begin{math}
        1+2=2+1=3
    \end{math}。
    
    \section{上下标}
    \subsection{上标}
    $3x^{55}+2x=16$

    $3x^{3x^{20}-x+2}=3$
    \subsection{下标}
    $a_0,a_2$

    $a_{100},a_{200}$
    \section{希腊字母}
    \begin{math}
        \alpha
        \beta
        \gamma
        \epsilon
        \pi
        \omega
        \Gamma
        \Delta
        \Theta
        \Pi
        \Omega
    \end{math}

    \section{数学函数}
    \begin{math}
        \log
        \sin
        \cos
        \arccos
        \arcsin
        \ln
    \end{math}

    $\sqrt[4]{x^2+5}$

    $y=log_2 x$

    \section{分式}
    $3/4$
    $\frac{3x}{x^3}$

    \section{行间公式}
    \subsection{美元符号}
    $$a+b=b+a$$如$$a+2=2+1=3$$
    \subsection{中括号}
    \[a+b=b+a\]如\[1+2=2+1=3\]。
    \subsection{displaymath环境}
    \begin{displaymath}
        a+b=b+a
    \end{displaymath}
    \subsection{自动编号公式equation环境}
    交换律见\ref{eq:hongdeyao}
    \begin{equation}
        a+b=b+a  \label{eq:hongdeyao}
    \end{equation}
    \subsection{不编号公式}
    %\begin{equation*}
    %    a+b=b+a
    %\end{equation*}


    {\kaishu 详细版本请见链接https://blog.csdn.net/NSJim/article/details/109045914}
\end{document}