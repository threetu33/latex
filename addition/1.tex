\documentclass[UTF8]{article}
% 取自文章https://blog.csdn.net/NSJim/article/details/109066847?ops_request_misc=%257B%2522request%255Fid%2522%253A%2522170683895116800185832519%2522%252C%2522scm%2522%253A%252220140713.130102334..%2522%257D&request_id=170683895116800185832519&biz_id=0&utm_medium=distribute.pc_search_result.none-task-blog-2~all~top_positive~default-1-109066847-null-null.142^v99^pc_search_result_base7&utm_term=latex&spm=1018.2226.3001.4187
\usepackage{ctex}

% 使用indentfirst宏包
\usepackage{indentfirst}
% 设置首行缩进距离
\setlength{\parindent}{2em}
% 单段取消缩进 \noindent
% 全局取消缩进 \setlength{\parindent}{0pt}
% 单段进行缩进 \hspace*{2em}段落\\

% 设置页面的环境,a4纸张大小,左右上下边距信息
\usepackage[a4paper,left=10mm,right=10mm,top=15mm,bottom=15mm]{geometry}

\title{这是一个\LaTeX{}语法的补充知识}
\author{洪德垚}
\date{\today}



\iffalse
中文支持 : ctex
latex数学公式支持 : amsmath
插入图片 : graphicx
算法排版 : algorithm algorithmic
插入代码块 : listings
\fi

\iffalse
换行 \\
offset代表增加的行间距 \\[offset]
换段 \par
新页 \newpage
\+字符 可以用转义字符使一些特殊字符强制显示
\fi

\iffalse
推荐:
{\songti 宋体}
{\kaishu 楷书}
{\fangsong 仿宋}
{\heiti 黑体}

{\bf 粗体}
{\it 斜体}
{\sl 斜体}
\fi


\begin{document}
    \maketitle   %添加这一行才能显示标题信息

    % 生成目录设置
    \renewcommand{\contentsname}{目录} %将content转为目录
    \tableofcontents

    \begin{abstract}
        该部分内容是放置摘要信息的。该部分内容是放置摘要信息的。该部分内容是放置摘要信息的。该部分内容是放置摘要信息的。该部分内容是放置摘要信息的。
    \end{abstract}

    西游记\footnote{中国古典四大名著之一}小说开头写道:
    \begin{quote}
    {\kaishu 东胜神洲有一花果山,山顶一石,受日月精华,生出一石猴。之后因为成功闯入水帘洞,被花果山诸猴拜为“美猴王”。}
    \end{quote}


    % 标题开始
    \section{一级标题1}
    第一段一级标题下的内容,一级标题下的内容,一级标题下的内容,一级标题下的内容,一级标题下的内容,一级标题下的内容,一级标题下的内容,一级标题下的内容。\par
    第二段一级标题下的内容,一级标题下的内容,一级标题下的内容,一级标题下的内容,一级标题下的内容,一级标题下的内容,一级标题下的内容,一级标题下的内容。

    \subsection{二级标题1.1}
    二级标题下的内容。

    \subsubsection{三级标题下的内容1.1.1}
    三级标题下的内容。

    \section{一级标题2}
    一级标题2中的内容


\end{document}